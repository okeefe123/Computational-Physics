\documentclass[10pt]{article}
\usepackage[margin=.7in]{geometry}
\usepackage{amsthm}
\usepackage{hyperref}
\usepackage{graphicx}
\usepackage{booktabs}
\listfiles

\begin{document}
 
\begin{center}
\large
\hfill Okeefe Niemann\\
\hfill 5/9/2014\\
\hfill 1281465\\
\hfill PHYS 115 \\
\LARGE \textbf{Assignment 5}\\
\end{center}

\normalsize
%%%%%%0%%%%%%%%%%%%%%%%%%%%%%%%%%%%%%%%%%%%%%%%%%%%%%%%%%%%%%%%%%%%%%%%%%%%%%%%%%%%%%%%%%%%%%%%%%%%%%%%%%%%%%
\section*{Question 1}
\subsection*{Part a}
\textbf{Comment:} See handwritten page on back for analytic solution.

\begin{itemize}
\subsection*{Part b}
\item Input:
\begin{verbatim}
#include <stdio.h>
#include <math.h>
#include <stdlib.h>


int main()
{
        int N = 12;
        int M = 10000;
        int i, j, k;
        double x[N], Y, moment;

for(k = 1; k < 7; k ++)
{
        moment = 0;
        
        for( j = 0; j < M; j++)
        {
                Y = 0;

                for(i = 0; i < N; i++)
                {
                        x[i] = rand() / (RAND_MAX + 1.0);

                        Y += sqrt(12 / N) * (x[i] - 0.5);
                }

                moment += pow(Y, k) / M;
         }
         
         printf("<X^%d> = %f\n", k, moment);
}

return 0;
}
\end{verbatim}
\item Output:
\begin{verbatim}
<X^1> = 0.001541
<X^2> = 0.970864
<X^3> = 0.051299
<X^4> = 3.102080
<X^5> = 0.112035
<X^6> = 14.217256
\end{verbatim}

%%%%%%%%%%%%%%%%%%%%%%%%%%%%%%%%%%%%%%%%%%%%%%%%%%%%%%%%%%%%%%%%%%%%%%%%%%%%%%%%%%%%%%%%%%%%%%%%%%%%%%%%%%
\section*{Question 2}
\subsection*{Part a/c}
See handwritten page for analytic solution

\subsection*{Part b}
\item Input:
\begin{verbatim}
#include <stdio.h>
#include <math.h>
#include <stdlib.h>
#include <time.h>

int main()
{
        int i, N = 100000;
        double x[N], value[N], prob, ab;
        srand(time(NULL));
        
        prob = 0;
        
        for(i = 0; i < N; i++)
        {
                x[i] = rand() / (RAND_MAX + 1.0);
                
                value[i] = tan(M_PI * x[i]);

                if(value[i] < 1.0 && value[i] > -1.0)
                        prob += 1.0 / N;
        }

        printf("probability |x| < 1: %f percent\n", prob);
}
\end{verbatim}
\item Output:
\begin{verbatim}
probability |x| < 1: 0.500420 percent
\end{verbatim}
\textbf{Comment:} Compared with the exact answer calculated on the back sheet ($\frac{1}/{2}$), the numerical answer has precision $10^{-3}$.

%%%%%%%%%%%%%%%%%%%%%%%%%%%%%%%%%%%%%%%%%%%%%%%%%%%%%%%%%%%%%%%%%%%%%%%%%%%%%%%%%%%%%%%%%%%%%%%%%%%%%%%%%%%
\section*{Question 3}
\item Input:
\begin{verbatim}
#include <stdio.h>
#include <time.h>
#include <math.h>
#include <stdlib.h>

//defines function to be integrated
double f(double x)
{
        return log(x);
}

int main()
{
        double x, a, b, exact, favg, f2avg, error, nsigma;
        int i, NPOINTS;
        
        NPOINTS = 100000;
        exact = log(4.0) - 1.0;                 //exact answer
        b = 2.0;                                //upper bound
        a = 1.0;                                //lower bound
        srand(time(NULL));                      //initializes random number generator 

        favg = 0;
        f2avg = 0;
        
        for(i = 0; i < NPOINTS; i++)
        {
                x = a + (b - a) * rand() / RAND_MAX;	       //computes random values for x
                favg += f(x);                                  //computes f(x)
                f2avg += f(x) * f(x);                          //computes f(x)^2
        }

        favg /= NPOINTS;
        f2avg /= NPOINTS;
        
        error = sqrt((f2avg - favg*favg)/ (NPOINTS - 1));    //error bar, or standard deviation

        printf("number of points = %d\n", NPOINTS);
        printf("analytic answer = %f\n", exact);
        printf("computational answer = %f\n", (b - a) * favg);
        printf("error bar = %f\n", error);
        printf("# standard deviations = %fsigma\n", fabs((favg - exact) / error));

return 0;
}
\end{verbatim}
\item Output:
\begin{verbatim}
number of points = 100000
analytic answer = 0.386294
computational answer = 0.385842
error bar = 0.000625
# standard deviations = 0.724393sigma
\end{verbatim}
\textbf{Comment:} The exact answer was about $0.724\sigma$ away from the answer calculated by Monte Carlo integration, meaning that the error is acceptable for our purposes. While the standard deviation seems to remain consistent as the number of points increases, the error bar dramatically decreases and our computational answer becomes increasingly more accurate. 

%%%%%%%%%%%%%%%%%%%%%%%%%%%%%%%%%%%%%%%%%%%%%%%%%%%%%%%%%%%%%%%%%%%%%%%%%%%%%%%%%%%%%%%%%%%%%%%%%%%%%%%%%%%%
\section*{Question 4}
\subsection*{Part a}
\textbf{Comment:} See attached handwritten page for analytic solution.

\subsection*{Part b}
\item Input:
\begin{verbatim}
#include <stdio.h>
#include <time.h>
#include <math.h>
#include <stdlib.h>

int main()
{
        double x[10], a, b, exact, favg, f2avg, error, nsigma;
        int i, j, NPOINTS;
        
        NPOINTS = 100000;
        exact = 155 / 6.0;      //exact answer
        b = 1.0;					//upper bound
        a = 0;					//lower bound
        srand(time(NULL));			//initializes random number generator 

        favg = 0;
        f2avg = 0;

        for(i = 0; i < NPOINTS; i++)
        {
                for(j = 0; j < 10; j++)
                {
                        x[j] = a + (b - a) * rand() / RAND_MAX;    //computes random values for x
                }
                
         //computes the values for <f> and <f>^2
         favg += pow(x[0] + x[1] + x[2] + x[3] + x[4] + x[5] + x[6] + x[7] + x[8] + x[9], 2);
         f2avg += pow(pow(x[0] + x[1] + x[2] + x[3] + x[4] + x[5] + x[6] + x[7] + x[8] + x[9], 2),2);
        }

        favg /= NPOINTS;
        f2avg /= NPOINTS;
        
        error = sqrt((f2avg - favg*favg)/ (NPOINTS - 1)); //error bar
        
        printf("number of points = %d\n", NPOINTS);
        printf("analytic answer = %f\n", exact);
        printf("computational answer = %f\n", (b - a) * favg);
        printf("error bar = %f\n", error);
        printf("# standard deviations = %fsigma\n", fabs((favg - exact) / error));

return 0;
}
\end{verbatim}
\item Output:
\begin{verbatim}
number of points = 100000
analytic answer = 25.833333
computational answer = 25.819427
error bar = 0.029112
# standard deviations = 0.477689sigma
\end{verbatim}

\textbf{Comment:} The number of standard deviations refer to how many values of $\sigma$ away the exact answer is from the above Monte Carlo calculation.

%%%%%%%%%%%%%%%%%%%%%%%%%%%%%%%%%%%%%%%%%%%%%%%%%%%%%%%%%%%%%%%%%%%%%%%%%%%%%%%%%%%%%%%%%%%%%%%%%%%%%%%%%%%%%%%%%%%%%
\section*{Question 5}
\item Input:
\begin{verbatim}
#include <stdio.h>
#include <math.h>
#include <time.h>
#include <stdlib.h>

int main()
{
        int i, t, NUMSTEPS;
        srand(time(NULL));
        float array[10] = {10, 50, 100, 200, 300, 400, 500, 1000, 10000, 100000};
        double xavg[10], x2avg[10], epsilon;
        
        for( i = 0; i < 10; i ++)
        {
                NUMSTEPS = array[i];

                for(t = 0; t < NUMSTEPS; t++)
                {
                        epsilon = rand()/(RAND_MAX + 1.0);
                        
                        if(epsilon <= 0.5)
                                epsilon = -1.0;
                        else
                                epsilon = 1.0;

                        xavg[i] += epsilon;
                        
                        x2avg[i] += epsilon * epsilon;

                }
        }
        
        printf("# Steps       <x>          <x^2>\n");

        for(i = 0; i < 10; i++)
        {
                printf("%.0f        %.1f    %.1f          %.0f\n", array[i], xavg[i], xavg[i] / array[i], x2avg[i]);
        }

return 0;	
}
\end{verbatim}
\item Output:
\begin{verbatim}
# Steps       <x>         <x>/Steps          <x^2>
10            4.0          0.400000         10
50            2.0          0.040000         50
100           0.0          0.000000         100
200         -16.0         -0.080000         200
500          14.0          0.028000         500
1000         -4.0         -0.004000         1000
10000       246.0          0.024600         10000
100000      -48.0         -0.000480         100000
1000000    -438.0         -0.000438         1000000
10000000  -3102.0         -0.000310         10000000
\end{verbatim}
\textbf{Comment:} Though  $\textless x \textgreater$ increases, ratio of $\textless x \textgreater$ to the number of steps steadily converges to 0.
\end{itemize}
\end{document}