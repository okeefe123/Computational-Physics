\documentclass[10pt]{article}
\usepackage[margin=.7in]{geometry}
\usepackage{amsthm}
\usepackage{hyperref}
\usepackage{graphicx}
\usepackage{booktabs}
\listfiles

\begin{document}
 
\begin{center}
\large
\hfill Okeefe Niemann\\
\hfill 5/5/2014\\
\hfill 1281465\\
\hfill PHYS 115 \\
\LARGE \textbf{Ising Model by Monte Carlo}\\
\end{center}
\normalsize
\section{Introduction}
In this project, I will be using the Monte Carlo technique to create a two dimensional ising model, extracting the average of the square of the magnetization for each temperature. By doing so, I will prove how the theoretical values of magnetization tend to zero as the temperature of the system reaches its critical value.

When a molecular system is connected to a heat reservoir, the state of the system can be statistically represented by:

$$P(n) = \f
\end{document}
